
% This LaTeX was auto-generated from an M-file by MATLAB.
% To make changes, update the M-file and republish this document.

\documentclass{article}
\usepackage{graphicx}
\usepackage{color}

\sloppy
\definecolor{lightgray}{gray}{0.5}
\setlength{\parindent}{0pt}

\begin{document}

    
    
\subsection*{Contents}

\begin{itemize}
\setlength{\itemsep}{-1ex}
   \item Segundo Proyecto An�lisis Aplicado
   \item Par�metros de Entrada y Salida
   \item Paths que necesitaremos a lo largo del programa:
   \item Automatizar creaci�n del archivo *.mod
   \item Automatizar creaci�n del archivo *.nl
   \item Encontrar las Betas
   \item Comparar con el test y los resultados originales
\end{itemize}


\subsection*{Segundo Proyecto An�lisis Aplicado}

\begin{verbatim}
% Dado cualquier base de datos estandarizada del formato:
%
% Identificador      VarIndep       VarDep1       VarDep2 ...
%
% 984727             1              .9867         -2.5446 ...
%
% 984728             1              .9313          0.2342 ...
%
% 984729             0              .3354         -3.4521 ...
%
% este programa realiza la regresi�n log�stica regularizada penalizando
% a la funci�n objetivo con la norma 2 de las betas multiplicada por una
% constante
%
% Es decir la funci�n objetivo queda como sigue:
%
% min -sum(yi*(xi'Beta))+ sum(log(1+exp(x1'Beta))) + gamma/2*norm(Beta)^2
%
% Podemos correr este ejemplo para probar la funci�n:
%
% [x1,porc]=regL(5,.5,.1, 30,0,'/bin/AMPL/');
%
function [betas, porc]=regL (gamma,corte,tamTest, nvar,reor,pathAmpl);
\end{verbatim}


\subsection*{Par�metros de Entrada y Salida}

\begin{verbatim}
% [ENTRADA] gamma = valor para regularizar
%
% [ENTRADA] corte = el valor de probabilidad para el cual aceptaremos entre (0,1)
%
% [ENTRADA] tamTrain = porcentaje que queremos que sea de test
%
% [ENTRADA] nvar = el numero de variables independientes
%
% [ENTRADA] reor = para sacar la muestra aleatoriamente (0=no aleat,1=aleat)
%
% [ENTRADA] pathAmpl = path de los programas de ampl
%
% [SALIDA] x1 = valor de las betas que toma el modelo
%
% [SALIDA] porc = porcentaje de aciertos que tuvo el modelo
%
\end{verbatim}


\subsection*{Paths que necesitaremos a lo largo del programa:}

\begin{verbatim}
%Path de ampl:
    path( path,'/bin/AMPL');
%Path de nuestro trabajo
    work=pwd;
\end{verbatim}


\subsection*{Automatizar creaci�n del archivo *.mod}

\begin{verbatim}
fil2 = fopen( 'LIST_large2', 'w' );
[train1,test1,ntrain1,ntest1] = wdbcData('Segundo',nvar,tamTest,reor);
fprintf('el tama�o del conjunto de entrenamiento es : %i \n',ntrain1)
fprintf('el tama�o del conjunto de test es : %i \n\n',ntest1)
estDatos(train1,'trainReor1.mod',gamma);
fprintf('Se ha creado el archivo trainReor1.mod en la direcci�n: \n %s \n\n',pwd);
fprintf(fil2,'trainReor1 \n');
fclose(fil2);
\end{verbatim}

        \color{lightgray} \begin{verbatim}el tama�o del conjunto de entrenamiento es : 512 
el tama�o del conjunto de test es : 57 

Se ha creado el archivo trainReor1.mod en la direcci�n: 
 /Users/SavrGG/Dropbox/Computadoras/001_MATLAB/001_AnalisisAplicado/006_Proyecto2 

\end{verbatim} \color{black}
    

\subsection*{Automatizar creaci�n del archivo *.nl}

\begin{verbatim}
writePerl(work);
perl('rCUTE');
fprintf('Se ha creado el archivo trainReor1.nl en la direcci�n: \n %s \n\n',pwd);
use_ampl_stub trainReor1.nl;
\end{verbatim}

        \color{lightgray} \begin{verbatim}Se ha creado el archivo trainReor1.nl en la direcci�n: 
 /Users/SavrGG/Dropbox/Computadoras/001_MATLAB/001_AnalisisAplicado/006_Proyecto2 

\end{verbatim} \color{black}
    

\subsection*{Encontrar las Betas}

\begin{verbatim}
betas=Newton1pruebas ('amplpnt','amplstub', 'trainReor1', 50)
% Valor de la funci�n y gradiente en el �ptimo:
\end{verbatim}

        \color{lightgray} \begin{verbatim}
betas =

    2.0260
   -4.8345
    0.2017
   -3.3105
    0.1829
    1.1873
   -6.6502
    4.9073
    3.3270
   -0.7268
    1.4684
    4.0385
   -1.3002
   -2.2868
    5.1602
    0.7778
    2.7905
   -4.0470
    4.7538
   -0.9940
   -6.5278
    5.3618
    3.4039
    4.9252
    7.5667
   -0.1536
   -2.0334
    2.6261
   -0.2926
    2.1219
    4.2440

\end{verbatim} \color{black}
    

\subsection*{Comparar con el test y los resultados originales}

\begin{verbatim}
Orig=test1(:,1);
resul=zeros(ntest1,1);
test1(:,1)=ones(ntest1,1);
suma=0;
for i=1:ntest1
b=-betas'*test1(i,:)';
valor = 1/(1+exp(b));
    if valor >= corte
        resul(i)=1;
    end

    if(Orig(i)==resul(i))
        suma=suma+1;
    end
end

porc=suma/ntest1;
fprintf('El porcentaje de predicci�n es: %5.4f \n', porc)
\end{verbatim}

        \color{lightgray} \begin{verbatim}El porcentaje de predicci�n es: 1.0000 
\end{verbatim} \color{black}
    \begin{verbatim}
end
\end{verbatim}



\end{document}
    
